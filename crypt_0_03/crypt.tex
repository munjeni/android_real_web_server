\documentclass{book}
\usepackage{makeidx}
\usepackage{amssymb}
\def\union{\cup}
\def\intersect{\cap}
\def\getsrandom{\stackrel{\rm R}{\gets}}
\def\cross{\times}
\def\cat{\hspace{0.5em} \| \hspace{0.5em}}
\def\catn{$\|$}
\def\divides{\hspace{0.3em} | \hspace{0.3em}}
\def\nequiv{\not\equiv}
\def\approx{\raisebox{0.2ex}{\mbox{\small $\sim$}}}
\def\lcm{{\rm lcm}}
\def\gcd{{\rm gcd}}
\def\log{{\rm log}}
\def\ord{{\rm ord}}
\def\abs{{\mathit abs}}
\def\rep{{\mathit rep}}
\def\mod{{\mathit\ mod\ }}
\renewcommand{\pmod}[1]{\ ({\rm mod\ }{#1})}
\newcommand{\floor}[1]{\left\lfloor{#1}\right\rfloor}
\newcommand{\ceil}[1]{\left\lceil{#1}\right\rceil}
\def\Or{{\rm\ or\ }}
\def\And{{\rm\ and\ }}
\def\iff{\hspace{1em}\Longleftrightarrow\hspace{1em}}
\def\implies{\Rightarrow}
\def\undefined{{\rm ``undefined"}}
\def\Proof{\vspace{1ex}\noindent {\bf Proof:}\hspace{1em}}
\let\oldphi\phi
\def\phi{\varphi}
\def\Pr{{\rm Pr}}
\newcommand{\str}[1]{{\mathbf{#1}}}
\def\F{{\mathbb F}}
\def\N{{\mathbb N}}
\def\Z{{\mathbb Z}}
\def\R{{\mathbb R}}
\def\C{{\mathbb C}}
\def\Q{{\mathbb Q}}

\newcommand{\url}[1]{\mbox{$<${#1}$>$}}
\newcommand{\emailaddr}[1]{\mbox{$<${#1}$>$}}
\def\twiddle{\raisebox{0.3ex}{\mbox{\tiny $\sim$}}}

\def\gap{\vspace{0.5ex}}
\makeindex
\begin{document}
\title{Tiny Crypto Library \\ Version 00.03}
\author{Tom St Denis \\
Algonquin College \\
\\
tomstdenis@yahoo.com \\
http://tomstdenis.home.dhs.org \\
111 Banning Rd \\
Kanata, Ontario \\
K2L 1C3 \\
Canada 
}
\maketitle
\newpage
\tableofcontents
\chapter{Introduction}
\section{What is libcrypt.a}
Since I really could care less for a name you can call this library either ``Tiny Crypto Library'' or ``libcrypt.a''.  I
think there is already a ``libcrypt.a'' with ``cygwin'' but I could care less about that too.

So what is ``libcrypt.a'' or just ``lba'' for short?  It is a portable ANSI C cryptographic library that supports
symmetric ciphers, one-way hashes, pseudo-random number generators and public key crypto (via RSA).  It is designed 
to compile out of the box with the GNU C Compiler (GCC) version 2.95.3 and higher.  

The library is designed so new ciphers/hashes/prngs can be added and the existing API (and helper API functions) will 
be able to use the new designs automatically.  There exist self-check functions for each cipher and hash to ensure that
they compile and execute to the design specifications.  The library also performs extensive parameter error checking
and will give verbose error messages.

\section{License}

All of the code except for ``mpi.c'', ``mpi-config.h'' and ``mpi.h'' has been written from scratch by Tom St Denis
and as such is under the official binding ``Tom Doesn't Care About Licenses'' or TDCAL license.  This entitles the 
user to use this library for {\em whatever} they want without any form of repayment to the author whatsoever.  Similarly
MPI is public domain software copyright'ed by Michael Fromberger and may be used.

Essentially, ``have fun kids.''

\section{Building the library}

To build the library on a GCC equipped platform simply type ``make'' at your command prompt.  It will build the library
file ``libcrypt.a''.  If you are on a non-x86 platform comment out the appropriate line in ``makefile''.  To build the
test executable just take off the ``.exe''.  I hope to have a makefile for *NIX platforms shortly.

To build the library on a non-GCC platform just build all of the source files (except test.c) and archive them in a 
single library.

To install the library copy all of the ``.h'' files into your include path and the single libcrypt.a file into your 
LIB path.

Et voila.

\chapter{The API}
\section{Introduction}

In general the API is very simple to memorize and use.  Most of the functions return either {\bf void} or {\bf int}.  Functions
that return {\bf int} will return either {\bf CRYPT\_OK} or {\bf CRYPT\_ERROR}.  If there is an error the character pointer
{\bf crypt\_error}\footnote{Note that it is lower case.} will be set.  

There is no initialization routine for the library and for the most part the code is thread safe.  The only thread
related issue is if you use the same symmetric cipher, hash or public key state data in multiple threads.  Normally
that is not an issue, therefore, the library is thread safe.

To include the prototypes for libcrypt into your own program simply include ``crypt.h'' like so:

\begin{verbatim}
#include <crypt.h>
int main(void) {
    return 0;
}
\end{verbatim}

\section{Macros}

There are a few helper macros to make the coding process a bit easier.  The first set are related to loading and storing
32/64-bit words in little endian format.  The macros are:

\begin{center}
\begin{tabular}{|c|c|c|}
     \hline STORE32L(x, y) & {\bf unsigned long} x, {\bf unsigned char} *y & $x \to y[0 \ldots 3]$ \\
     \hline STORE64L(x, y) & {\bf unsigned long long} x, {\bf unsigned char} *y & $x \to y[0 \ldots 7]$ \\
     \hline LOAD32L(x, y) & {\bf unsigned long} x, {\bf unsigned char} *y & $y[0 \ldots 3] \to x$ \\
     \hline LOAD64L(x, y) & {\bf unsigned long long} x, {\bf unsigned char} *y & $y[0 \ldots 7] \to x$ \\
     \hline BSWAP(x) & {\bf unsigned long} x & Swaps the byte order of x. \\
     \hline
\end{tabular}
\end{center}

There are 32-bit cyclic rotations as well:
\begin{center}
\begin{tabular}{|c|c|c|}
     \hline ROL(x, y) & {\bf unsigned long} x, {\bf unsigned long} y & $x << y$ \\
     \hline ROR(x, y) & {\bf unsigned long} x, {\bf unsigned long} y & $x >> y$ \\
     \hline
\end{tabular}
\end{center}

\section{Functions with Variable Length Output}
Certain functions such as ``rsa\_export()'' give an output that is variable length.  To prevent buffer overflows you
must pass it the length of the buffer\footnote{Extensive error checking is not in place but it will be in future releases so it is a good idea to follow through with these guidelines.} where
the output will be stored.  For example:

\begin{verbatim}
#include <crypt.h>
int main(void) {
    struct rsa_key key;
    unsigned char buffer[1024];
    int x;

    /* ... Make up the key somehow */

    /* lets export the key */ 
    x = 1024; 
    if (rsa_export(buffer, &x, PK_PUBLIC, &key) == CRYPT_ERROR) {
       printf("Export error: %s\n", crypt_error);
       return -1;
    }

    /* ... do something with the buffer */
    
    return 0;
}
\end{verbatim}

\chapter{Symmetric Block Ciphers}
\section{Core Functions}

libcrypt provides several block ciphers all in a plain vanila ECB block mode.  All ciphers store their scheduled keys in
a single union called ``symmetric\_key''.  This allows all ciphers to have the same prototype and store their keys as 
naturally as possible.  All ciphers provide four visible functions which are (given that XXX is the name of the cipher):
\begin{verbatim}
int XXX_setup(unsigned char *key, int keylen, int rounds, 
              union symmetric_key *skey);
\end{verbatim}

The XXX\_setup() routine will setup the cipher to be used with a given number of rounds and a given key length (in bytes).  
The number of rounds can be set to zero to use the default.  It returns {\bf CRYPT\_ERROR} if the parameters are 
invalid such as an incorrect key size or number of rounds.  If the cipher is setup correctly it returns {\bf CRYPT\_OK}.

\begin{verbatim}
void XXX_ecb_encrypt(unsigned char *pt, unsigned char *ct,
                     union symmetric_key *skey);

void XXX_ecb_decrypt(unsigned char *ct, unsigned char *pt,
                     union symmetric_key *skey);
\end{verbatim}

These two functions will encrypt or decrypt (respectively) a single block of text\footnote{The size of which depends on which cipher you are using.}
 and store the result where you want it.  It is possible that the input and output buffer are the same buffer.

\begin{verbatim}
int XXX_test(void);
\end{verbatim}

This function will return {\bf CRYPT\_OK} if the cipher matches the test vectors from the designs.  It returns 
{\bf CRYPT\_ERROR} if it fails to meet the test vectors.

An example snippet that encodes a block with Blowfish in ECB mode is below.

\begin{verbatim}
#include <crypt.h>
int main(void)
{ 
    unsigned char pt[8], ct[8], key[8];
    union symmetric_key skey;
    int x;

    /* make up a key */
    for (x = 0; x < 8; x++) pt[x] = key[x] = x;

    /* schedule the key */
    if (blowfish_setup(key, 8, 0, &skey) == CRYPT_ERROR) {
       printf("Setup error: %s\n", crypt_error);
       return -1;
    }    

    /* encrypt the block */
    blowfish_ecb_encrypt(pt, ct, &skey);
 
    /* decrypt the block */
    blowfish_ecb_decrypt(ct, pt, &skey);

    return 0;
}
\end{verbatim}

\section{The Cipher Descriptors}

To facilitate automatic routines an array of cipher descriptors is provided in the array ``cipher\_descriptor''.  An element
of this array has the following format:

\begin{verbatim}
struct _cipher_descriptor {
   char *name;
   int  min_key_length, max_key_length, 
        block_length, default_rounds;
   int  (*setup)      (unsigned char *key, int keylength, 
                       int num_rounds, union symmetric_key *skey);
   void (*ecb_encrypt)(unsigned char *pt, unsigned char *ct, 
                       union symmetric_key *key);
   void (*ecb_decrypt)(unsigned char *ct, unsigned char *pt, 
                       union symmetric_key *key);
   int (*test)        (void);
};
\end{verbatim}

Where ``name'' is the lower case ASCII version of the name.  The fields ``min\_key\_length'', ``max\_key\_length'' and
``block\_length'' are all the number of bytes not bits.  As a good rule of thumb it is assumed that the cipher supports
the min and max key lengths but not always everything in between.  The ``default\_rounds'' field is the default number
of rounds that will be used.

The remaining fields are all pointers to the core functions for each cipher.  The end of the cipher\_descriptor array is
marked when ``name'' equals {\bf NULL}.

As of this release the current cipher\_descriptors elements are

\begin{center}
\begin{tabular}{|c|c|c|c|c|}
     \hline Name & Block Size & Key Range & Rounds \\
     \hline Blowfish, ``blowfish'' & 8 & 8 ... 56 & 16 \\
     \hline RC5-32/12/b, ``rc5'' & 8 & 8 ... 128 & 12 ... 24 \\
     \hline RC6-32/20/b, ``rc6'' & 16 & 8 ... 128 & 20 \\
     \hline SAFER+, ``safer+'' & 16 & 16, 24, 32 & 8, 12, 16 \\
     \hline Serpent, ``serpent'' & 16 & 16 .. 32 & 32 \\
     \hline
\end{tabular}
\end{center}

Note that ``SAFER+'' does not allow the user to override the number of rounds.  The number of rounds varies per 
key length.  There are 8 rounds for 16 byte keys, 12 rounds for 24 byte keys and 16 rounds for 32 byte keys.  You can either
specify the correct number of rounds or zero and let it pick.

To work with the cipher\_descriptor array there is a function ``int find\_cipher({\bf char *}name)'' that will search
for a given name in the array.  It returns negative one if the cipher is not found, otherwise it returns the location
in the array where the cipher was found.  For example, to indirectly setup Blowfish you can also use:

\begin{small}
\begin{verbatim}
cipher_descriptor[find_cipher("blowfish")].setup(key, keylen, rounds, &skey);
\end{verbatim}
\end{small}

A good safety would be to check the return value before accessing the desired function.

\section{CBC and CTR Block Modes}
The library provides simple support routines for handling CBC and CTR encoded messages.  Assuming the mode you want
is XXX there is a structure called ``{\bf struct symmetric\_XXX}'' that will contain the information required to
use that mode.  They have virtually identical setup routines:
\begin{verbatim}
int cbc_start(int cipher, unsigned char *IV, unsigned char *key, 
              int keylen, int num_rounds, struct symmetric_CBC *cbc);

int ctr_start(int cipher, unsigned char *count, unsigned char *key, 
              int keylen, int num_rounds, struct symmetric_CTR *ctr);
\end{verbatim}

In both cases ``cipher'' is the index into the cipher\_descriptor array of the cipher you want to use.  The ``IV'' and
``count'' values are the initialization vectors.  You must fill them yourself and it is assumed they are the same length
as the block size of the cipher you choose.  The parameters ``key'', ``keylen'' and ``num\_rounds'' are the same as in
the XXX\_setup() function call.  The final parameter is a pointer to the structure you want to hold the information for
the mode of operation.

Both routines return {\bf CRYPT\_OK} if the cipher initialized correctly, otherwise they return {\bf CRYPT\_ERROR}.  To 
actually encrypt or decrypt the following routines are provided:
\begin{verbatim}
void cbc_encrypt(unsigned char *pt, unsigned char *ct, 
                 struct symmetric_CBC *cbc);
void cbc_decrypt(unsigned char *ct, unsigned char *pt,
                 struct symmetric_CBC *cbc);

void ctr_encrypt(unsigned char *pt, unsigned char *ct, 
                 int len, struct symmetric_CTR *ctr);
void ctr_decrypt(unsigned char *ct, unsigned char *pt, 
                 int len, struct symmetric_CTR *ctr);
\end{verbatim}

These routines work much like the ECB mode routines.  In the CBC case it always encrypts blocks of a size determined
by the cipher.  This means in CBC mode you should pad your message (with zeroes will work) so the message length
is a multiple of block cipher length.  In CTR mode you can specify the size of each block which means when you reach
the end of the message you can simply pass the value of the remaining length.

To decrypt in either mode you simply setup the CBC or CTR buffer like before (recall you have to fetch the IV value you used)
and use the decrypt routine on all of the blocks.

\chapter{One-Way Hash Functions}
\section{Core Functions}

Like the ciphers there are hash core functions and a universal structure to hold the hash state called 
``{\bf union hash\_state}''.  To initialize hash XXX (where XXX is the name) call:
\begin{verbatim}
void XXX_init(union hash_state *md);
\end{verbatim}

This simply sets up the hash to the default state governed by the specifications of the hash.  To add data to the 
message being hashed call:
\begin{verbatim}
void XXX_process(unsigned char *in, int len, 
                 union hash_state *md);
\end{verbatim}

Essentially all hash messages are virtually infinitely\footnote{All hashes are limited to $2^{32}$ bits} long message which is 
buffered.  You pass pieces of the data at a time to the process function.  To finally get the result of the hash
call:
\begin{verbatim}
void XXX_done(union hash_state *md, 
              unsigned char *out);
\end{verbatim}

This function will finish up the hash and store the result in the ``out'' array.  You must ensure that ``out'' is long
enough for the hash in question.  To test a hash function call:
\begin{verbatim}
int XXX_test(void);
\end{verbatim}

This will return {\bf CRYPTO\_OK} if the hash matches the test vectors, otherwise it returns {\bf CRYPTO\_ERROR}.  An
example snippet that hashes a message with md5 is given below.
\begin{verbatim}
#include <crypt.h>
int main(void)
{
    union hash_state md;
    unsigned char *in = "hello world", out[16];

    /* setup the hash */
    md5_init(&md);

    /* add the message */
    md5_process(in, strlen(in), &md);

    /* get the hash */
    md5_done(&md, out);

    return 0;
}
\end{verbatim}

\section{Hash Descriptors}

Like the set of ciphers the set of hashes have descriptors too.  They are stored in an array called ``hash\_descriptor'' and
are defined by:
\begin{verbatim}
struct _hash_descriptor {
    char *name;
    int hashsize;
    void (*init)   (union hash_state *);
    void (*process)(union hash_state *, 
                    unsigned char *, int);
    void (*done)   (union hash_state *, 
                    unsigned char *);
    int  (*test)   (void);
};
\end{verbatim}

Similarly ``name'' is the name of the hash function in ASCII (all lowercase).  ``hashsize'' is the size of the digest output
in bytes.  The remaining fields are pointers to the functions that do the respective tasks.  There is a function to
search the array as well called ``int find\_hash(char *name)''.  It returns -1 if the hash is not found, otherwise the
position in the descriptor table of the hash.

There are two helper functions as well:
\begin{verbatim}
int hash_memory(int hash, unsigned char *data, 
                int len, unsigned char *dst);

int hash_file(int hash, char *fname, 
              unsigned char *dst);
\end{verbatim}

Both functions return {\bf CRYPT\_OK} on success, otherwise they return {\bf CRYPT\_ERROR}.  The ``hash'' parameter is
the location in the descriptor table of the hash.  The functions are otherwise straightforward.  To perform the above 
hash with md5 the following code could be used:
\begin{verbatim}
hash_memory(find_hash("md5"), "hello world", 11, out);
\end{verbatim}

The following hashes are provided as of this release:
\begin{center}
\begin{tabular}{|c|c|}
      \hline Name & Size \\
      \hline SHA-256, ``sha256'' & 32 \\
      \hline SHA-1, ``sha1'' & 20 \\
      \hline MD5, ``md5'' & 16 \\
      \hline TIGER-192, ``tiger'' & 24 \\
      \hline
\end{tabular}
\end{center}

\chapter{Pseudo-Random Number Generators}
\section{Core Functions}

The library provides an array of core functions for PRNGs as well.  There is a universal structure called 
``{\bf union prng\_state}''.  To initialize a PRNG call:
\begin{verbatim}
int XXX_start(union prng_state *prng);
\end{verbatim}

This will setup the PRNG but not seed it.  Returns {\bf CRYPTO\_ERROR} if there is an error.  In order for the PRNG to be 
cryptographically useful you must give it entropy.  Ideally you'd have some OS level source to tap like in UNIX.  Since 
this is a portable library I leave it upto you to figure this one out.  Nyah Nyah!.  To add entropy to the PRNG call:
\begin{verbatim}
int XXX_add_entropy(unsigned char *in, int len, 
                    union prng_state *prng);
\end{verbatim}

Which returns {\bf CRYPTO\_OK} if the entropy was accepted.  Once you think you have enough entropy you call another
function to put the entropy into action.
\begin{verbatim}
int XXX_ready(union prng_state *prng);
\end{verbatim}

Which returns {\bf CRYPTO\_OK} if it is ready.  Finally to actually read bytes call:
\begin{verbatim}
int XXX_read(unsigned char *out, int len,
             union prng_state *prng);
\end{verbatim}

Which returns the number of bytes read from the PRNG.

\subsection{Remarks}

It is possible to be adding entropy and reading from a PRNG at the same time.  For example, if you first seed the PRNG
and call ready() you can now read from it.  You can also keep adding new entropy to it.  The new entropy will not be used
in the PRNG until ready() is called.  This allows the PRNG to be used and re-seeded at the same time.

No real error checking is guaranteed to see if the entropy is sufficient or if the PRNG is even in a ready state.  Simple
snippet to read 10 bytes from yarrow is below.

\begin{verbatim}
#include <crypt.h>
int main(void)
{
   union prng_state prng;
   unsigned char buf[10];

   /* start it */
   yarrow_start(&prng);

   /* add entropy */
   yarrow_add_entropy("hello world", 11, &prng);
 
   /* ready and read */
   yarrow_ready(&prng);
   yarrow_read(buf, 10, &prng);

   return 0;
}
\end{verbatim}

\section{PRNG Descriptors}

PRNGs have descriptors too (surprised?). Stored in the structure ``prng\_descriptor''.  The format of an element is:
\begin{verbatim}
struct _prng_descriptor {
    char *name;
    int (*start)      (union prng_state *);
    int (*add_entropy)(unsigned char *, int, 
                       union prng_state *);
    int (*ready)      (union prng_state *);
    int (*read)       (unsigned char *, int len, 
                       union prng_state *);
};
\end{verbatim}

There is a ``int find\_prng(char *name)'' function as well.  Returns -1 if the PRNG is not found, otherwise it returns
the position in the prng\_descriptor array.

There are no additional support routines for PRNGs.  Yarrow is the only currently support PRNG available.  It uses by 
default SHA1 and Blowfish but you can change that after calling ``yarrow\_start()''.

\chapter{RSA Routines}
\section{Core Functions}

For RSA routines a single ``struct rsa\_key'' structure is used.  To make a new RSA key call:
\begin{verbatim}
int rsa_make_key(union prng_state *prng, 
                 int wprng, int size, 
                 long e, struct rsa_key *key);
\end{verbatim}

Where ``wprng'' is the index into the PRNG descriptor array.  ``size'' is the size in bytes of the RSA modulus desired.
``e'' is the encryption exponent desired, typical values are 3, 17, 257 and 65537.  I suggest you stick with 65537 since its big
enough to prevent trivial math attacks and not super slow.  ``key'' is where the key is placed.  This routine returns
{\bf CRYPTO\_ERROR} if it fails to make a RSA key.

Todo raw work with the RSA function call:
\begin{verbatim}
int rsa_exptmod(unsigned char *in, int inlen, 
                unsigned char *out, int *outlen, 
                int which, struct rsa_key *key);
\end{verbatim}
This loads the bignum from ``in'' as a little endian word, raises it to either ``e'' or ``d'' and stores the result
in ``out'' and the size of the result in ``outlen''.  You must ensure that ``outlen'' is set to the maximum size of
the output buffer before calling this function.  This is to prevent buffer overruns.  ``which'' is set to 
{\bf PK\_PUBLIC} to use ``e'' (i.e. for encryption) and set to {\bf PK\_PRIVATE} to use ``d'' as the exponent.  This
function returns {\bf CRYPT\_ERROR} on error.

\section{Packet Routines}
The remaining RSA functions are non-standard but should (to the best of my knowledge) be secure if used correctly.  To
encrypt a buffer of memory in a hybrid fashion call:
\begin{verbatim}
int rsa_encrypt(unsigned char *in, int len, 
                unsigned char *out, int *outlen,
                union prng_state *prng, int wprng, 
                int cipher, struct rsa_key *key);
\end{verbatim}
This will encrypt the message with the cipher specified by ``cipher'' under a random key made by a PRNG specified by
``wprng'' and RSA encrypt the symmetric key with ``key''.  This stores all the relevent information in ``out'' and sets
the length in ``outlen''.  You must ensure that ``outlen'' is set to the buffer size before calling this.  This returns
{\bf CRYPT\_ERROR} on error.  To decrypt packets made by this routine call:
\begin{verbatim}
int rsa_decrypt(unsigned char *in, int len, 
                unsigned char *out, int *outlen, 
                struct rsa_key *key);
\end{verbatim}
Which works akin to rsa\_encrypt().  Similarly to sign/verify there are:
\begin{verbatim}
int rsa_sign(unsigned char *in, int inlen, 
             unsigned char *out, int *outlen, int hash, 
             union prng_state *prng, int wprng, struct rsa_key *key);

int rsa_verify(unsigned char *sig, 
               unsigned char *msg, 
               int inlen, int *stat, 
               struct rsa_key *key);
\end{verbatim}

The verify function sets ``stat'' to 1 if it passes or to 0 if it fails.  The ``sig'' parameter is the output of the
rsa\_sign() function and ``msg'' is the original msg that was signed.  To import/export RSA keys as a memory buffer (
e.g. to store them to disk) call:
\begin{verbatim}
int rsa_export(unsigned char *out, int *outlen, 
               int type, struct rsa_key *key);

int rsa_import(unsigned char *in, struct rsa_key *key);
\end{verbatim}

To free the memory used by an RSA key call:
\begin{verbatim}
void rsa_free(struct rsa_key *key);
\end{verbatim}

\backmatter
\appendix
\documentclass{book}
\usepackage{makeidx}
\usepackage{amssymb}
\def\union{\cup}
\def\intersect{\cap}
\def\getsrandom{\stackrel{\rm R}{\gets}}
\def\cross{\times}
\def\cat{\hspace{0.5em} \| \hspace{0.5em}}
\def\catn{$\|$}
\def\divides{\hspace{0.3em} | \hspace{0.3em}}
\def\nequiv{\not\equiv}
\def\approx{\raisebox{0.2ex}{\mbox{\small $\sim$}}}
\def\lcm{{\rm lcm}}
\def\gcd{{\rm gcd}}
\def\log{{\rm log}}
\def\ord{{\rm ord}}
\def\abs{{\mathit abs}}
\def\rep{{\mathit rep}}
\def\mod{{\mathit\ mod\ }}
\renewcommand{\pmod}[1]{\ ({\rm mod\ }{#1})}
\newcommand{\floor}[1]{\left\lfloor{#1}\right\rfloor}
\newcommand{\ceil}[1]{\left\lceil{#1}\right\rceil}
\def\Or{{\rm\ or\ }}
\def\And{{\rm\ and\ }}
\def\iff{\hspace{1em}\Longleftrightarrow\hspace{1em}}
\def\implies{\Rightarrow}
\def\undefined{{\rm ``undefined"}}
\def\Proof{\vspace{1ex}\noindent {\bf Proof:}\hspace{1em}}
\let\oldphi\phi
\def\phi{\varphi}
\def\Pr{{\rm Pr}}
\newcommand{\str}[1]{{\mathbf{#1}}}
\def\F{{\mathbb F}}
\def\N{{\mathbb N}}
\def\Z{{\mathbb Z}}
\def\R{{\mathbb R}}
\def\C{{\mathbb C}}
\def\Q{{\mathbb Q}}

\newcommand{\url}[1]{\mbox{$<${#1}$>$}}
\newcommand{\emailaddr}[1]{\mbox{$<${#1}$>$}}
\def\twiddle{\raisebox{0.3ex}{\mbox{\tiny $\sim$}}}

\def\gap{\vspace{0.5ex}}
\makeindex
\begin{document}
\title{Tiny Crypto Library \\ Version 00.03}
\author{Tom St Denis \\
Algonquin College \\
\\
tomstdenis@yahoo.com \\
http://tomstdenis.home.dhs.org \\
111 Banning Rd \\
Kanata, Ontario \\
K2L 1C3 \\
Canada 
}
\maketitle
\newpage
\tableofcontents
\chapter{Introduction}
\section{What is libcrypt.a}
Since I really could care less for a name you can call this library either ``Tiny Crypto Library'' or ``libcrypt.a''.  I
think there is already a ``libcrypt.a'' with ``cygwin'' but I could care less about that too.

So what is ``libcrypt.a'' or just ``lba'' for short?  It is a portable ANSI C cryptographic library that supports
symmetric ciphers, one-way hashes, pseudo-random number generators and public key crypto (via RSA).  It is designed 
to compile out of the box with the GNU C Compiler (GCC) version 2.95.3 and higher.  

The library is designed so new ciphers/hashes/prngs can be added and the existing API (and helper API functions) will 
be able to use the new designs automatically.  There exist self-check functions for each cipher and hash to ensure that
they compile and execute to the design specifications.  The library also performs extensive parameter error checking
and will give verbose error messages.

\section{License}

All of the code except for ``mpi.c'', ``mpi-config.h'' and ``mpi.h'' has been written from scratch by Tom St Denis
and as such is under the official binding ``Tom Doesn't Care About Licenses'' or TDCAL license.  This entitles the 
user to use this library for {\em whatever} they want without any form of repayment to the author whatsoever.  Similarly
MPI is public domain software copyright'ed by Michael Fromberger and may be used.

Essentially, ``have fun kids.''

\section{Building the library}

To build the library on a GCC equipped platform simply type ``make'' at your command prompt.  It will build the library
file ``libcrypt.a''.  If you are on a non-x86 platform comment out the appropriate line in ``makefile''.  To build the
test executable just take off the ``.exe''.  I hope to have a makefile for *NIX platforms shortly.

To build the library on a non-GCC platform just build all of the source files (except test.c) and archive them in a 
single library.

To install the library copy all of the ``.h'' files into your include path and the single libcrypt.a file into your 
LIB path.

Et voila.

\chapter{The API}
\section{Introduction}

In general the API is very simple to memorize and use.  Most of the functions return either {\bf void} or {\bf int}.  Functions
that return {\bf int} will return either {\bf CRYPT\_OK} or {\bf CRYPT\_ERROR}.  If there is an error the character pointer
{\bf crypt\_error}\footnote{Note that it is lower case.} will be set.  

There is no initialization routine for the library and for the most part the code is thread safe.  The only thread
related issue is if you use the same symmetric cipher, hash or public key state data in multiple threads.  Normally
that is not an issue, therefore, the library is thread safe.

To include the prototypes for libcrypt into your own program simply include ``crypt.h'' like so:

\begin{verbatim}
#include <crypt.h>
int main(void) {
    return 0;
}
\end{verbatim}

\section{Macros}

There are a few helper macros to make the coding process a bit easier.  The first set are related to loading and storing
32/64-bit words in little endian format.  The macros are:

\begin{center}
\begin{tabular}{|c|c|c|}
     \hline STORE32L(x, y) & {\bf unsigned long} x, {\bf unsigned char} *y & $x \to y[0 \ldots 3]$ \\
     \hline STORE64L(x, y) & {\bf unsigned long long} x, {\bf unsigned char} *y & $x \to y[0 \ldots 7]$ \\
     \hline LOAD32L(x, y) & {\bf unsigned long} x, {\bf unsigned char} *y & $y[0 \ldots 3] \to x$ \\
     \hline LOAD64L(x, y) & {\bf unsigned long long} x, {\bf unsigned char} *y & $y[0 \ldots 7] \to x$ \\
     \hline BSWAP(x) & {\bf unsigned long} x & Swaps the byte order of x. \\
     \hline
\end{tabular}
\end{center}

There are 32-bit cyclic rotations as well:
\begin{center}
\begin{tabular}{|c|c|c|}
     \hline ROL(x, y) & {\bf unsigned long} x, {\bf unsigned long} y & $x << y$ \\
     \hline ROR(x, y) & {\bf unsigned long} x, {\bf unsigned long} y & $x >> y$ \\
     \hline
\end{tabular}
\end{center}

\section{Functions with Variable Length Output}
Certain functions such as ``rsa\_export()'' give an output that is variable length.  To prevent buffer overflows you
must pass it the length of the buffer\footnote{Extensive error checking is not in place but it will be in future releases so it is a good idea to follow through with these guidelines.} where
the output will be stored.  For example:

\begin{verbatim}
#include <crypt.h>
int main(void) {
    struct rsa_key key;
    unsigned char buffer[1024];
    int x;

    /* ... Make up the key somehow */

    /* lets export the key */ 
    x = 1024; 
    if (rsa_export(buffer, &x, PK_PUBLIC, &key) == CRYPT_ERROR) {
       printf("Export error: %s\n", crypt_error);
       return -1;
    }

    /* ... do something with the buffer */
    
    return 0;
}
\end{verbatim}

\chapter{Symmetric Block Ciphers}
\section{Core Functions}

libcrypt provides several block ciphers all in a plain vanila ECB block mode.  All ciphers store their scheduled keys in
a single union called ``symmetric\_key''.  This allows all ciphers to have the same prototype and store their keys as 
naturally as possible.  All ciphers provide four visible functions which are (given that XXX is the name of the cipher):
\begin{verbatim}
int XXX_setup(unsigned char *key, int keylen, int rounds, 
              union symmetric_key *skey);
\end{verbatim}

The XXX\_setup() routine will setup the cipher to be used with a given number of rounds and a given key length (in bytes).  
The number of rounds can be set to zero to use the default.  It returns {\bf CRYPT\_ERROR} if the parameters are 
invalid such as an incorrect key size or number of rounds.  If the cipher is setup correctly it returns {\bf CRYPT\_OK}.

\begin{verbatim}
void XXX_ecb_encrypt(unsigned char *pt, unsigned char *ct,
                     union symmetric_key *skey);

void XXX_ecb_decrypt(unsigned char *ct, unsigned char *pt,
                     union symmetric_key *skey);
\end{verbatim}

These two functions will encrypt or decrypt (respectively) a single block of text\footnote{The size of which depends on which cipher you are using.}
 and store the result where you want it.  It is possible that the input and output buffer are the same buffer.

\begin{verbatim}
int XXX_test(void);
\end{verbatim}

This function will return {\bf CRYPT\_OK} if the cipher matches the test vectors from the designs.  It returns 
{\bf CRYPT\_ERROR} if it fails to meet the test vectors.

An example snippet that encodes a block with Blowfish in ECB mode is below.

\begin{verbatim}
#include <crypt.h>
int main(void)
{ 
    unsigned char pt[8], ct[8], key[8];
    union symmetric_key skey;
    int x;

    /* make up a key */
    for (x = 0; x < 8; x++) pt[x] = key[x] = x;

    /* schedule the key */
    if (blowfish_setup(key, 8, 0, &skey) == CRYPT_ERROR) {
       printf("Setup error: %s\n", crypt_error);
       return -1;
    }    

    /* encrypt the block */
    blowfish_ecb_encrypt(pt, ct, &skey);
 
    /* decrypt the block */
    blowfish_ecb_decrypt(ct, pt, &skey);

    return 0;
}
\end{verbatim}

\section{The Cipher Descriptors}

To facilitate automatic routines an array of cipher descriptors is provided in the array ``cipher\_descriptor''.  An element
of this array has the following format:

\begin{verbatim}
struct _cipher_descriptor {
   char *name;
   int  min_key_length, max_key_length, 
        block_length, default_rounds;
   int  (*setup)      (unsigned char *key, int keylength, 
                       int num_rounds, union symmetric_key *skey);
   void (*ecb_encrypt)(unsigned char *pt, unsigned char *ct, 
                       union symmetric_key *key);
   void (*ecb_decrypt)(unsigned char *ct, unsigned char *pt, 
                       union symmetric_key *key);
   int (*test)        (void);
};
\end{verbatim}

Where ``name'' is the lower case ASCII version of the name.  The fields ``min\_key\_length'', ``max\_key\_length'' and
``block\_length'' are all the number of bytes not bits.  As a good rule of thumb it is assumed that the cipher supports
the min and max key lengths but not always everything in between.  The ``default\_rounds'' field is the default number
of rounds that will be used.

The remaining fields are all pointers to the core functions for each cipher.  The end of the cipher\_descriptor array is
marked when ``name'' equals {\bf NULL}.

As of this release the current cipher\_descriptors elements are

\begin{center}
\begin{tabular}{|c|c|c|c|c|}
     \hline Name & Block Size & Key Range & Rounds \\
     \hline Blowfish, ``blowfish'' & 8 & 8 ... 56 & 16 \\
     \hline RC5-32/12/b, ``rc5'' & 8 & 8 ... 128 & 12 ... 24 \\
     \hline RC6-32/20/b, ``rc6'' & 16 & 8 ... 128 & 20 \\
     \hline SAFER+, ``safer+'' & 16 & 16, 24, 32 & 8, 12, 16 \\
     \hline Serpent, ``serpent'' & 16 & 16 .. 32 & 32 \\
     \hline
\end{tabular}
\end{center}

Note that ``SAFER+'' does not allow the user to override the number of rounds.  The number of rounds varies per 
key length.  There are 8 rounds for 16 byte keys, 12 rounds for 24 byte keys and 16 rounds for 32 byte keys.  You can either
specify the correct number of rounds or zero and let it pick.

To work with the cipher\_descriptor array there is a function ``int find\_cipher({\bf char *}name)'' that will search
for a given name in the array.  It returns negative one if the cipher is not found, otherwise it returns the location
in the array where the cipher was found.  For example, to indirectly setup Blowfish you can also use:

\begin{small}
\begin{verbatim}
cipher_descriptor[find_cipher("blowfish")].setup(key, keylen, rounds, &skey);
\end{verbatim}
\end{small}

A good safety would be to check the return value before accessing the desired function.

\section{CBC and CTR Block Modes}
The library provides simple support routines for handling CBC and CTR encoded messages.  Assuming the mode you want
is XXX there is a structure called ``{\bf struct symmetric\_XXX}'' that will contain the information required to
use that mode.  They have virtually identical setup routines:
\begin{verbatim}
int cbc_start(int cipher, unsigned char *IV, unsigned char *key, 
              int keylen, int num_rounds, struct symmetric_CBC *cbc);

int ctr_start(int cipher, unsigned char *count, unsigned char *key, 
              int keylen, int num_rounds, struct symmetric_CTR *ctr);
\end{verbatim}

In both cases ``cipher'' is the index into the cipher\_descriptor array of the cipher you want to use.  The ``IV'' and
``count'' values are the initialization vectors.  You must fill them yourself and it is assumed they are the same length
as the block size of the cipher you choose.  The parameters ``key'', ``keylen'' and ``num\_rounds'' are the same as in
the XXX\_setup() function call.  The final parameter is a pointer to the structure you want to hold the information for
the mode of operation.

Both routines return {\bf CRYPT\_OK} if the cipher initialized correctly, otherwise they return {\bf CRYPT\_ERROR}.  To 
actually encrypt or decrypt the following routines are provided:
\begin{verbatim}
void cbc_encrypt(unsigned char *pt, unsigned char *ct, 
                 struct symmetric_CBC *cbc);
void cbc_decrypt(unsigned char *ct, unsigned char *pt,
                 struct symmetric_CBC *cbc);

void ctr_encrypt(unsigned char *pt, unsigned char *ct, 
                 int len, struct symmetric_CTR *ctr);
void ctr_decrypt(unsigned char *ct, unsigned char *pt, 
                 int len, struct symmetric_CTR *ctr);
\end{verbatim}

These routines work much like the ECB mode routines.  In the CBC case it always encrypts blocks of a size determined
by the cipher.  This means in CBC mode you should pad your message (with zeroes will work) so the message length
is a multiple of block cipher length.  In CTR mode you can specify the size of each block which means when you reach
the end of the message you can simply pass the value of the remaining length.

To decrypt in either mode you simply setup the CBC or CTR buffer like before (recall you have to fetch the IV value you used)
and use the decrypt routine on all of the blocks.

\chapter{One-Way Hash Functions}
\section{Core Functions}

Like the ciphers there are hash core functions and a universal structure to hold the hash state called 
``{\bf union hash\_state}''.  To initialize hash XXX (where XXX is the name) call:
\begin{verbatim}
void XXX_init(union hash_state *md);
\end{verbatim}

This simply sets up the hash to the default state governed by the specifications of the hash.  To add data to the 
message being hashed call:
\begin{verbatim}
void XXX_process(unsigned char *in, int len, 
                 union hash_state *md);
\end{verbatim}

Essentially all hash messages are virtually infinitely\footnote{All hashes are limited to $2^{32}$ bits} long message which is 
buffered.  You pass pieces of the data at a time to the process function.  To finally get the result of the hash
call:
\begin{verbatim}
void XXX_done(union hash_state *md, 
              unsigned char *out);
\end{verbatim}

This function will finish up the hash and store the result in the ``out'' array.  You must ensure that ``out'' is long
enough for the hash in question.  To test a hash function call:
\begin{verbatim}
int XXX_test(void);
\end{verbatim}

This will return {\bf CRYPTO\_OK} if the hash matches the test vectors, otherwise it returns {\bf CRYPTO\_ERROR}.  An
example snippet that hashes a message with md5 is given below.
\begin{verbatim}
#include <crypt.h>
int main(void)
{
    union hash_state md;
    unsigned char *in = "hello world", out[16];

    /* setup the hash */
    md5_init(&md);

    /* add the message */
    md5_process(in, strlen(in), &md);

    /* get the hash */
    md5_done(&md, out);

    return 0;
}
\end{verbatim}

\section{Hash Descriptors}

Like the set of ciphers the set of hashes have descriptors too.  They are stored in an array called ``hash\_descriptor'' and
are defined by:
\begin{verbatim}
struct _hash_descriptor {
    char *name;
    int hashsize;
    void (*init)   (union hash_state *);
    void (*process)(union hash_state *, 
                    unsigned char *, int);
    void (*done)   (union hash_state *, 
                    unsigned char *);
    int  (*test)   (void);
};
\end{verbatim}

Similarly ``name'' is the name of the hash function in ASCII (all lowercase).  ``hashsize'' is the size of the digest output
in bytes.  The remaining fields are pointers to the functions that do the respective tasks.  There is a function to
search the array as well called ``int find\_hash(char *name)''.  It returns -1 if the hash is not found, otherwise the
position in the descriptor table of the hash.

There are two helper functions as well:
\begin{verbatim}
int hash_memory(int hash, unsigned char *data, 
                int len, unsigned char *dst);

int hash_file(int hash, char *fname, 
              unsigned char *dst);
\end{verbatim}

Both functions return {\bf CRYPT\_OK} on success, otherwise they return {\bf CRYPT\_ERROR}.  The ``hash'' parameter is
the location in the descriptor table of the hash.  The functions are otherwise straightforward.  To perform the above 
hash with md5 the following code could be used:
\begin{verbatim}
hash_memory(find_hash("md5"), "hello world", 11, out);
\end{verbatim}

The following hashes are provided as of this release:
\begin{center}
\begin{tabular}{|c|c|}
      \hline Name & Size \\
      \hline SHA-256, ``sha256'' & 32 \\
      \hline SHA-1, ``sha1'' & 20 \\
      \hline MD5, ``md5'' & 16 \\
      \hline TIGER-192, ``tiger'' & 24 \\
      \hline
\end{tabular}
\end{center}

\chapter{Pseudo-Random Number Generators}
\section{Core Functions}

The library provides an array of core functions for PRNGs as well.  There is a universal structure called 
``{\bf union prng\_state}''.  To initialize a PRNG call:
\begin{verbatim}
int XXX_start(union prng_state *prng);
\end{verbatim}

This will setup the PRNG but not seed it.  Returns {\bf CRYPTO\_ERROR} if there is an error.  In order for the PRNG to be 
cryptographically useful you must give it entropy.  Ideally you'd have some OS level source to tap like in UNIX.  Since 
this is a portable library I leave it upto you to figure this one out.  Nyah Nyah!.  To add entropy to the PRNG call:
\begin{verbatim}
int XXX_add_entropy(unsigned char *in, int len, 
                    union prng_state *prng);
\end{verbatim}

Which returns {\bf CRYPTO\_OK} if the entropy was accepted.  Once you think you have enough entropy you call another
function to put the entropy into action.
\begin{verbatim}
int XXX_ready(union prng_state *prng);
\end{verbatim}

Which returns {\bf CRYPTO\_OK} if it is ready.  Finally to actually read bytes call:
\begin{verbatim}
int XXX_read(unsigned char *out, int len,
             union prng_state *prng);
\end{verbatim}

Which returns the number of bytes read from the PRNG.

\subsection{Remarks}

It is possible to be adding entropy and reading from a PRNG at the same time.  For example, if you first seed the PRNG
and call ready() you can now read from it.  You can also keep adding new entropy to it.  The new entropy will not be used
in the PRNG until ready() is called.  This allows the PRNG to be used and re-seeded at the same time.

No real error checking is guaranteed to see if the entropy is sufficient or if the PRNG is even in a ready state.  Simple
snippet to read 10 bytes from yarrow is below.

\begin{verbatim}
#include <crypt.h>
int main(void)
{
   union prng_state prng;
   unsigned char buf[10];

   /* start it */
   yarrow_start(&prng);

   /* add entropy */
   yarrow_add_entropy("hello world", 11, &prng);
 
   /* ready and read */
   yarrow_ready(&prng);
   yarrow_read(buf, 10, &prng);

   return 0;
}
\end{verbatim}

\section{PRNG Descriptors}

PRNGs have descriptors too (surprised?). Stored in the structure ``prng\_descriptor''.  The format of an element is:
\begin{verbatim}
struct _prng_descriptor {
    char *name;
    int (*start)      (union prng_state *);
    int (*add_entropy)(unsigned char *, int, 
                       union prng_state *);
    int (*ready)      (union prng_state *);
    int (*read)       (unsigned char *, int len, 
                       union prng_state *);
};
\end{verbatim}

There is a ``int find\_prng(char *name)'' function as well.  Returns -1 if the PRNG is not found, otherwise it returns
the position in the prng\_descriptor array.

There are no additional support routines for PRNGs.  Yarrow is the only currently support PRNG available.  It uses by 
default SHA1 and Blowfish but you can change that after calling ``yarrow\_start()''.

\chapter{RSA Routines}
\section{Core Functions}

For RSA routines a single ``struct rsa\_key'' structure is used.  To make a new RSA key call:
\begin{verbatim}
int rsa_make_key(union prng_state *prng, 
                 int wprng, int size, 
                 long e, struct rsa_key *key);
\end{verbatim}

Where ``wprng'' is the index into the PRNG descriptor array.  ``size'' is the size in bytes of the RSA modulus desired.
``e'' is the encryption exponent desired, typical values are 3, 17, 257 and 65537.  I suggest you stick with 65537 since its big
enough to prevent trivial math attacks and not super slow.  ``key'' is where the key is placed.  This routine returns
{\bf CRYPTO\_ERROR} if it fails to make a RSA key.

Todo raw work with the RSA function call:
\begin{verbatim}
int rsa_exptmod(unsigned char *in, int inlen, 
                unsigned char *out, int *outlen, 
                int which, struct rsa_key *key);
\end{verbatim}
This loads the bignum from ``in'' as a little endian word, raises it to either ``e'' or ``d'' and stores the result
in ``out'' and the size of the result in ``outlen''.  You must ensure that ``outlen'' is set to the maximum size of
the output buffer before calling this function.  This is to prevent buffer overruns.  ``which'' is set to 
{\bf PK\_PUBLIC} to use ``e'' (i.e. for encryption) and set to {\bf PK\_PRIVATE} to use ``d'' as the exponent.  This
function returns {\bf CRYPT\_ERROR} on error.

\section{Packet Routines}
The remaining RSA functions are non-standard but should (to the best of my knowledge) be secure if used correctly.  To
encrypt a buffer of memory in a hybrid fashion call:
\begin{verbatim}
int rsa_encrypt(unsigned char *in, int len, 
                unsigned char *out, int *outlen,
                union prng_state *prng, int wprng, 
                int cipher, struct rsa_key *key);
\end{verbatim}
This will encrypt the message with the cipher specified by ``cipher'' under a random key made by a PRNG specified by
``wprng'' and RSA encrypt the symmetric key with ``key''.  This stores all the relevent information in ``out'' and sets
the length in ``outlen''.  You must ensure that ``outlen'' is set to the buffer size before calling this.  This returns
{\bf CRYPT\_ERROR} on error.  To decrypt packets made by this routine call:
\begin{verbatim}
int rsa_decrypt(unsigned char *in, int len, 
                unsigned char *out, int *outlen, 
                struct rsa_key *key);
\end{verbatim}
Which works akin to rsa\_encrypt().  Similarly to sign/verify there are:
\begin{verbatim}
int rsa_sign(unsigned char *in, int inlen, 
             unsigned char *out, int *outlen, int hash, 
             union prng_state *prng, int wprng, struct rsa_key *key);

int rsa_verify(unsigned char *sig, 
               unsigned char *msg, 
               int inlen, int *stat, 
               struct rsa_key *key);
\end{verbatim}

The verify function sets ``stat'' to 1 if it passes or to 0 if it fails.  The ``sig'' parameter is the output of the
rsa\_sign() function and ``msg'' is the original msg that was signed.  To import/export RSA keys as a memory buffer (
e.g. to store them to disk) call:
\begin{verbatim}
int rsa_export(unsigned char *out, int *outlen, 
               int type, struct rsa_key *key);

int rsa_import(unsigned char *in, struct rsa_key *key);
\end{verbatim}

To free the memory used by an RSA key call:
\begin{verbatim}
void rsa_free(struct rsa_key *key);
\end{verbatim}

\backmatter
\appendix
\documentclass{book}
\usepackage{makeidx}
\usepackage{amssymb}
\def\union{\cup}
\def\intersect{\cap}
\def\getsrandom{\stackrel{\rm R}{\gets}}
\def\cross{\times}
\def\cat{\hspace{0.5em} \| \hspace{0.5em}}
\def\catn{$\|$}
\def\divides{\hspace{0.3em} | \hspace{0.3em}}
\def\nequiv{\not\equiv}
\def\approx{\raisebox{0.2ex}{\mbox{\small $\sim$}}}
\def\lcm{{\rm lcm}}
\def\gcd{{\rm gcd}}
\def\log{{\rm log}}
\def\ord{{\rm ord}}
\def\abs{{\mathit abs}}
\def\rep{{\mathit rep}}
\def\mod{{\mathit\ mod\ }}
\renewcommand{\pmod}[1]{\ ({\rm mod\ }{#1})}
\newcommand{\floor}[1]{\left\lfloor{#1}\right\rfloor}
\newcommand{\ceil}[1]{\left\lceil{#1}\right\rceil}
\def\Or{{\rm\ or\ }}
\def\And{{\rm\ and\ }}
\def\iff{\hspace{1em}\Longleftrightarrow\hspace{1em}}
\def\implies{\Rightarrow}
\def\undefined{{\rm ``undefined"}}
\def\Proof{\vspace{1ex}\noindent {\bf Proof:}\hspace{1em}}
\let\oldphi\phi
\def\phi{\varphi}
\def\Pr{{\rm Pr}}
\newcommand{\str}[1]{{\mathbf{#1}}}
\def\F{{\mathbb F}}
\def\N{{\mathbb N}}
\def\Z{{\mathbb Z}}
\def\R{{\mathbb R}}
\def\C{{\mathbb C}}
\def\Q{{\mathbb Q}}

\newcommand{\url}[1]{\mbox{$<${#1}$>$}}
\newcommand{\emailaddr}[1]{\mbox{$<${#1}$>$}}
\def\twiddle{\raisebox{0.3ex}{\mbox{\tiny $\sim$}}}

\def\gap{\vspace{0.5ex}}
\makeindex
\begin{document}
\title{Tiny Crypto Library \\ Version 00.03}
\author{Tom St Denis \\
Algonquin College \\
\\
tomstdenis@yahoo.com \\
http://tomstdenis.home.dhs.org \\
111 Banning Rd \\
Kanata, Ontario \\
K2L 1C3 \\
Canada 
}
\maketitle
\newpage
\tableofcontents
\chapter{Introduction}
\section{What is libcrypt.a}
Since I really could care less for a name you can call this library either ``Tiny Crypto Library'' or ``libcrypt.a''.  I
think there is already a ``libcrypt.a'' with ``cygwin'' but I could care less about that too.

So what is ``libcrypt.a'' or just ``lba'' for short?  It is a portable ANSI C cryptographic library that supports
symmetric ciphers, one-way hashes, pseudo-random number generators and public key crypto (via RSA).  It is designed 
to compile out of the box with the GNU C Compiler (GCC) version 2.95.3 and higher.  

The library is designed so new ciphers/hashes/prngs can be added and the existing API (and helper API functions) will 
be able to use the new designs automatically.  There exist self-check functions for each cipher and hash to ensure that
they compile and execute to the design specifications.  The library also performs extensive parameter error checking
and will give verbose error messages.

\section{License}

All of the code except for ``mpi.c'', ``mpi-config.h'' and ``mpi.h'' has been written from scratch by Tom St Denis
and as such is under the official binding ``Tom Doesn't Care About Licenses'' or TDCAL license.  This entitles the 
user to use this library for {\em whatever} they want without any form of repayment to the author whatsoever.  Similarly
MPI is public domain software copyright'ed by Michael Fromberger and may be used.

Essentially, ``have fun kids.''

\section{Building the library}

To build the library on a GCC equipped platform simply type ``make'' at your command prompt.  It will build the library
file ``libcrypt.a''.  If you are on a non-x86 platform comment out the appropriate line in ``makefile''.  To build the
test executable just take off the ``.exe''.  I hope to have a makefile for *NIX platforms shortly.

To build the library on a non-GCC platform just build all of the source files (except test.c) and archive them in a 
single library.

To install the library copy all of the ``.h'' files into your include path and the single libcrypt.a file into your 
LIB path.

Et voila.

\chapter{The API}
\section{Introduction}

In general the API is very simple to memorize and use.  Most of the functions return either {\bf void} or {\bf int}.  Functions
that return {\bf int} will return either {\bf CRYPT\_OK} or {\bf CRYPT\_ERROR}.  If there is an error the character pointer
{\bf crypt\_error}\footnote{Note that it is lower case.} will be set.  

There is no initialization routine for the library and for the most part the code is thread safe.  The only thread
related issue is if you use the same symmetric cipher, hash or public key state data in multiple threads.  Normally
that is not an issue, therefore, the library is thread safe.

To include the prototypes for libcrypt into your own program simply include ``crypt.h'' like so:

\begin{verbatim}
#include <crypt.h>
int main(void) {
    return 0;
}
\end{verbatim}

\section{Macros}

There are a few helper macros to make the coding process a bit easier.  The first set are related to loading and storing
32/64-bit words in little endian format.  The macros are:

\begin{center}
\begin{tabular}{|c|c|c|}
     \hline STORE32L(x, y) & {\bf unsigned long} x, {\bf unsigned char} *y & $x \to y[0 \ldots 3]$ \\
     \hline STORE64L(x, y) & {\bf unsigned long long} x, {\bf unsigned char} *y & $x \to y[0 \ldots 7]$ \\
     \hline LOAD32L(x, y) & {\bf unsigned long} x, {\bf unsigned char} *y & $y[0 \ldots 3] \to x$ \\
     \hline LOAD64L(x, y) & {\bf unsigned long long} x, {\bf unsigned char} *y & $y[0 \ldots 7] \to x$ \\
     \hline BSWAP(x) & {\bf unsigned long} x & Swaps the byte order of x. \\
     \hline
\end{tabular}
\end{center}

There are 32-bit cyclic rotations as well:
\begin{center}
\begin{tabular}{|c|c|c|}
     \hline ROL(x, y) & {\bf unsigned long} x, {\bf unsigned long} y & $x << y$ \\
     \hline ROR(x, y) & {\bf unsigned long} x, {\bf unsigned long} y & $x >> y$ \\
     \hline
\end{tabular}
\end{center}

\section{Functions with Variable Length Output}
Certain functions such as ``rsa\_export()'' give an output that is variable length.  To prevent buffer overflows you
must pass it the length of the buffer\footnote{Extensive error checking is not in place but it will be in future releases so it is a good idea to follow through with these guidelines.} where
the output will be stored.  For example:

\begin{verbatim}
#include <crypt.h>
int main(void) {
    struct rsa_key key;
    unsigned char buffer[1024];
    int x;

    /* ... Make up the key somehow */

    /* lets export the key */ 
    x = 1024; 
    if (rsa_export(buffer, &x, PK_PUBLIC, &key) == CRYPT_ERROR) {
       printf("Export error: %s\n", crypt_error);
       return -1;
    }

    /* ... do something with the buffer */
    
    return 0;
}
\end{verbatim}

\chapter{Symmetric Block Ciphers}
\section{Core Functions}

libcrypt provides several block ciphers all in a plain vanila ECB block mode.  All ciphers store their scheduled keys in
a single union called ``symmetric\_key''.  This allows all ciphers to have the same prototype and store their keys as 
naturally as possible.  All ciphers provide four visible functions which are (given that XXX is the name of the cipher):
\begin{verbatim}
int XXX_setup(unsigned char *key, int keylen, int rounds, 
              union symmetric_key *skey);
\end{verbatim}

The XXX\_setup() routine will setup the cipher to be used with a given number of rounds and a given key length (in bytes).  
The number of rounds can be set to zero to use the default.  It returns {\bf CRYPT\_ERROR} if the parameters are 
invalid such as an incorrect key size or number of rounds.  If the cipher is setup correctly it returns {\bf CRYPT\_OK}.

\begin{verbatim}
void XXX_ecb_encrypt(unsigned char *pt, unsigned char *ct,
                     union symmetric_key *skey);

void XXX_ecb_decrypt(unsigned char *ct, unsigned char *pt,
                     union symmetric_key *skey);
\end{verbatim}

These two functions will encrypt or decrypt (respectively) a single block of text\footnote{The size of which depends on which cipher you are using.}
 and store the result where you want it.  It is possible that the input and output buffer are the same buffer.

\begin{verbatim}
int XXX_test(void);
\end{verbatim}

This function will return {\bf CRYPT\_OK} if the cipher matches the test vectors from the designs.  It returns 
{\bf CRYPT\_ERROR} if it fails to meet the test vectors.

An example snippet that encodes a block with Blowfish in ECB mode is below.

\begin{verbatim}
#include <crypt.h>
int main(void)
{ 
    unsigned char pt[8], ct[8], key[8];
    union symmetric_key skey;
    int x;

    /* make up a key */
    for (x = 0; x < 8; x++) pt[x] = key[x] = x;

    /* schedule the key */
    if (blowfish_setup(key, 8, 0, &skey) == CRYPT_ERROR) {
       printf("Setup error: %s\n", crypt_error);
       return -1;
    }    

    /* encrypt the block */
    blowfish_ecb_encrypt(pt, ct, &skey);
 
    /* decrypt the block */
    blowfish_ecb_decrypt(ct, pt, &skey);

    return 0;
}
\end{verbatim}

\section{The Cipher Descriptors}

To facilitate automatic routines an array of cipher descriptors is provided in the array ``cipher\_descriptor''.  An element
of this array has the following format:

\begin{verbatim}
struct _cipher_descriptor {
   char *name;
   int  min_key_length, max_key_length, 
        block_length, default_rounds;
   int  (*setup)      (unsigned char *key, int keylength, 
                       int num_rounds, union symmetric_key *skey);
   void (*ecb_encrypt)(unsigned char *pt, unsigned char *ct, 
                       union symmetric_key *key);
   void (*ecb_decrypt)(unsigned char *ct, unsigned char *pt, 
                       union symmetric_key *key);
   int (*test)        (void);
};
\end{verbatim}

Where ``name'' is the lower case ASCII version of the name.  The fields ``min\_key\_length'', ``max\_key\_length'' and
``block\_length'' are all the number of bytes not bits.  As a good rule of thumb it is assumed that the cipher supports
the min and max key lengths but not always everything in between.  The ``default\_rounds'' field is the default number
of rounds that will be used.

The remaining fields are all pointers to the core functions for each cipher.  The end of the cipher\_descriptor array is
marked when ``name'' equals {\bf NULL}.

As of this release the current cipher\_descriptors elements are

\begin{center}
\begin{tabular}{|c|c|c|c|c|}
     \hline Name & Block Size & Key Range & Rounds \\
     \hline Blowfish, ``blowfish'' & 8 & 8 ... 56 & 16 \\
     \hline RC5-32/12/b, ``rc5'' & 8 & 8 ... 128 & 12 ... 24 \\
     \hline RC6-32/20/b, ``rc6'' & 16 & 8 ... 128 & 20 \\
     \hline SAFER+, ``safer+'' & 16 & 16, 24, 32 & 8, 12, 16 \\
     \hline Serpent, ``serpent'' & 16 & 16 .. 32 & 32 \\
     \hline
\end{tabular}
\end{center}

Note that ``SAFER+'' does not allow the user to override the number of rounds.  The number of rounds varies per 
key length.  There are 8 rounds for 16 byte keys, 12 rounds for 24 byte keys and 16 rounds for 32 byte keys.  You can either
specify the correct number of rounds or zero and let it pick.

To work with the cipher\_descriptor array there is a function ``int find\_cipher({\bf char *}name)'' that will search
for a given name in the array.  It returns negative one if the cipher is not found, otherwise it returns the location
in the array where the cipher was found.  For example, to indirectly setup Blowfish you can also use:

\begin{small}
\begin{verbatim}
cipher_descriptor[find_cipher("blowfish")].setup(key, keylen, rounds, &skey);
\end{verbatim}
\end{small}

A good safety would be to check the return value before accessing the desired function.

\section{CBC and CTR Block Modes}
The library provides simple support routines for handling CBC and CTR encoded messages.  Assuming the mode you want
is XXX there is a structure called ``{\bf struct symmetric\_XXX}'' that will contain the information required to
use that mode.  They have virtually identical setup routines:
\begin{verbatim}
int cbc_start(int cipher, unsigned char *IV, unsigned char *key, 
              int keylen, int num_rounds, struct symmetric_CBC *cbc);

int ctr_start(int cipher, unsigned char *count, unsigned char *key, 
              int keylen, int num_rounds, struct symmetric_CTR *ctr);
\end{verbatim}

In both cases ``cipher'' is the index into the cipher\_descriptor array of the cipher you want to use.  The ``IV'' and
``count'' values are the initialization vectors.  You must fill them yourself and it is assumed they are the same length
as the block size of the cipher you choose.  The parameters ``key'', ``keylen'' and ``num\_rounds'' are the same as in
the XXX\_setup() function call.  The final parameter is a pointer to the structure you want to hold the information for
the mode of operation.

Both routines return {\bf CRYPT\_OK} if the cipher initialized correctly, otherwise they return {\bf CRYPT\_ERROR}.  To 
actually encrypt or decrypt the following routines are provided:
\begin{verbatim}
void cbc_encrypt(unsigned char *pt, unsigned char *ct, 
                 struct symmetric_CBC *cbc);
void cbc_decrypt(unsigned char *ct, unsigned char *pt,
                 struct symmetric_CBC *cbc);

void ctr_encrypt(unsigned char *pt, unsigned char *ct, 
                 int len, struct symmetric_CTR *ctr);
void ctr_decrypt(unsigned char *ct, unsigned char *pt, 
                 int len, struct symmetric_CTR *ctr);
\end{verbatim}

These routines work much like the ECB mode routines.  In the CBC case it always encrypts blocks of a size determined
by the cipher.  This means in CBC mode you should pad your message (with zeroes will work) so the message length
is a multiple of block cipher length.  In CTR mode you can specify the size of each block which means when you reach
the end of the message you can simply pass the value of the remaining length.

To decrypt in either mode you simply setup the CBC or CTR buffer like before (recall you have to fetch the IV value you used)
and use the decrypt routine on all of the blocks.

\chapter{One-Way Hash Functions}
\section{Core Functions}

Like the ciphers there are hash core functions and a universal structure to hold the hash state called 
``{\bf union hash\_state}''.  To initialize hash XXX (where XXX is the name) call:
\begin{verbatim}
void XXX_init(union hash_state *md);
\end{verbatim}

This simply sets up the hash to the default state governed by the specifications of the hash.  To add data to the 
message being hashed call:
\begin{verbatim}
void XXX_process(unsigned char *in, int len, 
                 union hash_state *md);
\end{verbatim}

Essentially all hash messages are virtually infinitely\footnote{All hashes are limited to $2^{32}$ bits} long message which is 
buffered.  You pass pieces of the data at a time to the process function.  To finally get the result of the hash
call:
\begin{verbatim}
void XXX_done(union hash_state *md, 
              unsigned char *out);
\end{verbatim}

This function will finish up the hash and store the result in the ``out'' array.  You must ensure that ``out'' is long
enough for the hash in question.  To test a hash function call:
\begin{verbatim}
int XXX_test(void);
\end{verbatim}

This will return {\bf CRYPTO\_OK} if the hash matches the test vectors, otherwise it returns {\bf CRYPTO\_ERROR}.  An
example snippet that hashes a message with md5 is given below.
\begin{verbatim}
#include <crypt.h>
int main(void)
{
    union hash_state md;
    unsigned char *in = "hello world", out[16];

    /* setup the hash */
    md5_init(&md);

    /* add the message */
    md5_process(in, strlen(in), &md);

    /* get the hash */
    md5_done(&md, out);

    return 0;
}
\end{verbatim}

\section{Hash Descriptors}

Like the set of ciphers the set of hashes have descriptors too.  They are stored in an array called ``hash\_descriptor'' and
are defined by:
\begin{verbatim}
struct _hash_descriptor {
    char *name;
    int hashsize;
    void (*init)   (union hash_state *);
    void (*process)(union hash_state *, 
                    unsigned char *, int);
    void (*done)   (union hash_state *, 
                    unsigned char *);
    int  (*test)   (void);
};
\end{verbatim}

Similarly ``name'' is the name of the hash function in ASCII (all lowercase).  ``hashsize'' is the size of the digest output
in bytes.  The remaining fields are pointers to the functions that do the respective tasks.  There is a function to
search the array as well called ``int find\_hash(char *name)''.  It returns -1 if the hash is not found, otherwise the
position in the descriptor table of the hash.

There are two helper functions as well:
\begin{verbatim}
int hash_memory(int hash, unsigned char *data, 
                int len, unsigned char *dst);

int hash_file(int hash, char *fname, 
              unsigned char *dst);
\end{verbatim}

Both functions return {\bf CRYPT\_OK} on success, otherwise they return {\bf CRYPT\_ERROR}.  The ``hash'' parameter is
the location in the descriptor table of the hash.  The functions are otherwise straightforward.  To perform the above 
hash with md5 the following code could be used:
\begin{verbatim}
hash_memory(find_hash("md5"), "hello world", 11, out);
\end{verbatim}

The following hashes are provided as of this release:
\begin{center}
\begin{tabular}{|c|c|}
      \hline Name & Size \\
      \hline SHA-256, ``sha256'' & 32 \\
      \hline SHA-1, ``sha1'' & 20 \\
      \hline MD5, ``md5'' & 16 \\
      \hline TIGER-192, ``tiger'' & 24 \\
      \hline
\end{tabular}
\end{center}

\chapter{Pseudo-Random Number Generators}
\section{Core Functions}

The library provides an array of core functions for PRNGs as well.  There is a universal structure called 
``{\bf union prng\_state}''.  To initialize a PRNG call:
\begin{verbatim}
int XXX_start(union prng_state *prng);
\end{verbatim}

This will setup the PRNG but not seed it.  Returns {\bf CRYPTO\_ERROR} if there is an error.  In order for the PRNG to be 
cryptographically useful you must give it entropy.  Ideally you'd have some OS level source to tap like in UNIX.  Since 
this is a portable library I leave it upto you to figure this one out.  Nyah Nyah!.  To add entropy to the PRNG call:
\begin{verbatim}
int XXX_add_entropy(unsigned char *in, int len, 
                    union prng_state *prng);
\end{verbatim}

Which returns {\bf CRYPTO\_OK} if the entropy was accepted.  Once you think you have enough entropy you call another
function to put the entropy into action.
\begin{verbatim}
int XXX_ready(union prng_state *prng);
\end{verbatim}

Which returns {\bf CRYPTO\_OK} if it is ready.  Finally to actually read bytes call:
\begin{verbatim}
int XXX_read(unsigned char *out, int len,
             union prng_state *prng);
\end{verbatim}

Which returns the number of bytes read from the PRNG.

\subsection{Remarks}

It is possible to be adding entropy and reading from a PRNG at the same time.  For example, if you first seed the PRNG
and call ready() you can now read from it.  You can also keep adding new entropy to it.  The new entropy will not be used
in the PRNG until ready() is called.  This allows the PRNG to be used and re-seeded at the same time.

No real error checking is guaranteed to see if the entropy is sufficient or if the PRNG is even in a ready state.  Simple
snippet to read 10 bytes from yarrow is below.

\begin{verbatim}
#include <crypt.h>
int main(void)
{
   union prng_state prng;
   unsigned char buf[10];

   /* start it */
   yarrow_start(&prng);

   /* add entropy */
   yarrow_add_entropy("hello world", 11, &prng);
 
   /* ready and read */
   yarrow_ready(&prng);
   yarrow_read(buf, 10, &prng);

   return 0;
}
\end{verbatim}

\section{PRNG Descriptors}

PRNGs have descriptors too (surprised?). Stored in the structure ``prng\_descriptor''.  The format of an element is:
\begin{verbatim}
struct _prng_descriptor {
    char *name;
    int (*start)      (union prng_state *);
    int (*add_entropy)(unsigned char *, int, 
                       union prng_state *);
    int (*ready)      (union prng_state *);
    int (*read)       (unsigned char *, int len, 
                       union prng_state *);
};
\end{verbatim}

There is a ``int find\_prng(char *name)'' function as well.  Returns -1 if the PRNG is not found, otherwise it returns
the position in the prng\_descriptor array.

There are no additional support routines for PRNGs.  Yarrow is the only currently support PRNG available.  It uses by 
default SHA1 and Blowfish but you can change that after calling ``yarrow\_start()''.

\chapter{RSA Routines}
\section{Core Functions}

For RSA routines a single ``struct rsa\_key'' structure is used.  To make a new RSA key call:
\begin{verbatim}
int rsa_make_key(union prng_state *prng, 
                 int wprng, int size, 
                 long e, struct rsa_key *key);
\end{verbatim}

Where ``wprng'' is the index into the PRNG descriptor array.  ``size'' is the size in bytes of the RSA modulus desired.
``e'' is the encryption exponent desired, typical values are 3, 17, 257 and 65537.  I suggest you stick with 65537 since its big
enough to prevent trivial math attacks and not super slow.  ``key'' is where the key is placed.  This routine returns
{\bf CRYPTO\_ERROR} if it fails to make a RSA key.

Todo raw work with the RSA function call:
\begin{verbatim}
int rsa_exptmod(unsigned char *in, int inlen, 
                unsigned char *out, int *outlen, 
                int which, struct rsa_key *key);
\end{verbatim}
This loads the bignum from ``in'' as a little endian word, raises it to either ``e'' or ``d'' and stores the result
in ``out'' and the size of the result in ``outlen''.  You must ensure that ``outlen'' is set to the maximum size of
the output buffer before calling this function.  This is to prevent buffer overruns.  ``which'' is set to 
{\bf PK\_PUBLIC} to use ``e'' (i.e. for encryption) and set to {\bf PK\_PRIVATE} to use ``d'' as the exponent.  This
function returns {\bf CRYPT\_ERROR} on error.

\section{Packet Routines}
The remaining RSA functions are non-standard but should (to the best of my knowledge) be secure if used correctly.  To
encrypt a buffer of memory in a hybrid fashion call:
\begin{verbatim}
int rsa_encrypt(unsigned char *in, int len, 
                unsigned char *out, int *outlen,
                union prng_state *prng, int wprng, 
                int cipher, struct rsa_key *key);
\end{verbatim}
This will encrypt the message with the cipher specified by ``cipher'' under a random key made by a PRNG specified by
``wprng'' and RSA encrypt the symmetric key with ``key''.  This stores all the relevent information in ``out'' and sets
the length in ``outlen''.  You must ensure that ``outlen'' is set to the buffer size before calling this.  This returns
{\bf CRYPT\_ERROR} on error.  To decrypt packets made by this routine call:
\begin{verbatim}
int rsa_decrypt(unsigned char *in, int len, 
                unsigned char *out, int *outlen, 
                struct rsa_key *key);
\end{verbatim}
Which works akin to rsa\_encrypt().  Similarly to sign/verify there are:
\begin{verbatim}
int rsa_sign(unsigned char *in, int inlen, 
             unsigned char *out, int *outlen, int hash, 
             union prng_state *prng, int wprng, struct rsa_key *key);

int rsa_verify(unsigned char *sig, 
               unsigned char *msg, 
               int inlen, int *stat, 
               struct rsa_key *key);
\end{verbatim}

The verify function sets ``stat'' to 1 if it passes or to 0 if it fails.  The ``sig'' parameter is the output of the
rsa\_sign() function and ``msg'' is the original msg that was signed.  To import/export RSA keys as a memory buffer (
e.g. to store them to disk) call:
\begin{verbatim}
int rsa_export(unsigned char *out, int *outlen, 
               int type, struct rsa_key *key);

int rsa_import(unsigned char *in, struct rsa_key *key);
\end{verbatim}

To free the memory used by an RSA key call:
\begin{verbatim}
void rsa_free(struct rsa_key *key);
\end{verbatim}

\backmatter
\appendix
\documentclass{book}
\usepackage{makeidx}
\usepackage{amssymb}
\def\union{\cup}
\def\intersect{\cap}
\def\getsrandom{\stackrel{\rm R}{\gets}}
\def\cross{\times}
\def\cat{\hspace{0.5em} \| \hspace{0.5em}}
\def\catn{$\|$}
\def\divides{\hspace{0.3em} | \hspace{0.3em}}
\def\nequiv{\not\equiv}
\def\approx{\raisebox{0.2ex}{\mbox{\small $\sim$}}}
\def\lcm{{\rm lcm}}
\def\gcd{{\rm gcd}}
\def\log{{\rm log}}
\def\ord{{\rm ord}}
\def\abs{{\mathit abs}}
\def\rep{{\mathit rep}}
\def\mod{{\mathit\ mod\ }}
\renewcommand{\pmod}[1]{\ ({\rm mod\ }{#1})}
\newcommand{\floor}[1]{\left\lfloor{#1}\right\rfloor}
\newcommand{\ceil}[1]{\left\lceil{#1}\right\rceil}
\def\Or{{\rm\ or\ }}
\def\And{{\rm\ and\ }}
\def\iff{\hspace{1em}\Longleftrightarrow\hspace{1em}}
\def\implies{\Rightarrow}
\def\undefined{{\rm ``undefined"}}
\def\Proof{\vspace{1ex}\noindent {\bf Proof:}\hspace{1em}}
\let\oldphi\phi
\def\phi{\varphi}
\def\Pr{{\rm Pr}}
\newcommand{\str}[1]{{\mathbf{#1}}}
\def\F{{\mathbb F}}
\def\N{{\mathbb N}}
\def\Z{{\mathbb Z}}
\def\R{{\mathbb R}}
\def\C{{\mathbb C}}
\def\Q{{\mathbb Q}}

\newcommand{\url}[1]{\mbox{$<${#1}$>$}}
\newcommand{\emailaddr}[1]{\mbox{$<${#1}$>$}}
\def\twiddle{\raisebox{0.3ex}{\mbox{\tiny $\sim$}}}

\def\gap{\vspace{0.5ex}}
\makeindex
\begin{document}
\title{Tiny Crypto Library \\ Version 00.03}
\author{Tom St Denis \\
Algonquin College \\
\\
tomstdenis@yahoo.com \\
http://tomstdenis.home.dhs.org \\
111 Banning Rd \\
Kanata, Ontario \\
K2L 1C3 \\
Canada 
}
\maketitle
\newpage
\tableofcontents
\chapter{Introduction}
\section{What is libcrypt.a}
Since I really could care less for a name you can call this library either ``Tiny Crypto Library'' or ``libcrypt.a''.  I
think there is already a ``libcrypt.a'' with ``cygwin'' but I could care less about that too.

So what is ``libcrypt.a'' or just ``lba'' for short?  It is a portable ANSI C cryptographic library that supports
symmetric ciphers, one-way hashes, pseudo-random number generators and public key crypto (via RSA).  It is designed 
to compile out of the box with the GNU C Compiler (GCC) version 2.95.3 and higher.  

The library is designed so new ciphers/hashes/prngs can be added and the existing API (and helper API functions) will 
be able to use the new designs automatically.  There exist self-check functions for each cipher and hash to ensure that
they compile and execute to the design specifications.  The library also performs extensive parameter error checking
and will give verbose error messages.

\section{License}

All of the code except for ``mpi.c'', ``mpi-config.h'' and ``mpi.h'' has been written from scratch by Tom St Denis
and as such is under the official binding ``Tom Doesn't Care About Licenses'' or TDCAL license.  This entitles the 
user to use this library for {\em whatever} they want without any form of repayment to the author whatsoever.  Similarly
MPI is public domain software copyright'ed by Michael Fromberger and may be used.

Essentially, ``have fun kids.''

\section{Building the library}

To build the library on a GCC equipped platform simply type ``make'' at your command prompt.  It will build the library
file ``libcrypt.a''.  If you are on a non-x86 platform comment out the appropriate line in ``makefile''.  To build the
test executable just take off the ``.exe''.  I hope to have a makefile for *NIX platforms shortly.

To build the library on a non-GCC platform just build all of the source files (except test.c) and archive them in a 
single library.

To install the library copy all of the ``.h'' files into your include path and the single libcrypt.a file into your 
LIB path.

Et voila.

\chapter{The API}
\section{Introduction}

In general the API is very simple to memorize and use.  Most of the functions return either {\bf void} or {\bf int}.  Functions
that return {\bf int} will return either {\bf CRYPT\_OK} or {\bf CRYPT\_ERROR}.  If there is an error the character pointer
{\bf crypt\_error}\footnote{Note that it is lower case.} will be set.  

There is no initialization routine for the library and for the most part the code is thread safe.  The only thread
related issue is if you use the same symmetric cipher, hash or public key state data in multiple threads.  Normally
that is not an issue, therefore, the library is thread safe.

To include the prototypes for libcrypt into your own program simply include ``crypt.h'' like so:

\begin{verbatim}
#include <crypt.h>
int main(void) {
    return 0;
}
\end{verbatim}

\section{Macros}

There are a few helper macros to make the coding process a bit easier.  The first set are related to loading and storing
32/64-bit words in little endian format.  The macros are:

\begin{center}
\begin{tabular}{|c|c|c|}
     \hline STORE32L(x, y) & {\bf unsigned long} x, {\bf unsigned char} *y & $x \to y[0 \ldots 3]$ \\
     \hline STORE64L(x, y) & {\bf unsigned long long} x, {\bf unsigned char} *y & $x \to y[0 \ldots 7]$ \\
     \hline LOAD32L(x, y) & {\bf unsigned long} x, {\bf unsigned char} *y & $y[0 \ldots 3] \to x$ \\
     \hline LOAD64L(x, y) & {\bf unsigned long long} x, {\bf unsigned char} *y & $y[0 \ldots 7] \to x$ \\
     \hline BSWAP(x) & {\bf unsigned long} x & Swaps the byte order of x. \\
     \hline
\end{tabular}
\end{center}

There are 32-bit cyclic rotations as well:
\begin{center}
\begin{tabular}{|c|c|c|}
     \hline ROL(x, y) & {\bf unsigned long} x, {\bf unsigned long} y & $x << y$ \\
     \hline ROR(x, y) & {\bf unsigned long} x, {\bf unsigned long} y & $x >> y$ \\
     \hline
\end{tabular}
\end{center}

\section{Functions with Variable Length Output}
Certain functions such as ``rsa\_export()'' give an output that is variable length.  To prevent buffer overflows you
must pass it the length of the buffer\footnote{Extensive error checking is not in place but it will be in future releases so it is a good idea to follow through with these guidelines.} where
the output will be stored.  For example:

\begin{verbatim}
#include <crypt.h>
int main(void) {
    struct rsa_key key;
    unsigned char buffer[1024];
    int x;

    /* ... Make up the key somehow */

    /* lets export the key */ 
    x = 1024; 
    if (rsa_export(buffer, &x, PK_PUBLIC, &key) == CRYPT_ERROR) {
       printf("Export error: %s\n", crypt_error);
       return -1;
    }

    /* ... do something with the buffer */
    
    return 0;
}
\end{verbatim}

\chapter{Symmetric Block Ciphers}
\section{Core Functions}

libcrypt provides several block ciphers all in a plain vanila ECB block mode.  All ciphers store their scheduled keys in
a single union called ``symmetric\_key''.  This allows all ciphers to have the same prototype and store their keys as 
naturally as possible.  All ciphers provide four visible functions which are (given that XXX is the name of the cipher):
\begin{verbatim}
int XXX_setup(unsigned char *key, int keylen, int rounds, 
              union symmetric_key *skey);
\end{verbatim}

The XXX\_setup() routine will setup the cipher to be used with a given number of rounds and a given key length (in bytes).  
The number of rounds can be set to zero to use the default.  It returns {\bf CRYPT\_ERROR} if the parameters are 
invalid such as an incorrect key size or number of rounds.  If the cipher is setup correctly it returns {\bf CRYPT\_OK}.

\begin{verbatim}
void XXX_ecb_encrypt(unsigned char *pt, unsigned char *ct,
                     union symmetric_key *skey);

void XXX_ecb_decrypt(unsigned char *ct, unsigned char *pt,
                     union symmetric_key *skey);
\end{verbatim}

These two functions will encrypt or decrypt (respectively) a single block of text\footnote{The size of which depends on which cipher you are using.}
 and store the result where you want it.  It is possible that the input and output buffer are the same buffer.

\begin{verbatim}
int XXX_test(void);
\end{verbatim}

This function will return {\bf CRYPT\_OK} if the cipher matches the test vectors from the designs.  It returns 
{\bf CRYPT\_ERROR} if it fails to meet the test vectors.

An example snippet that encodes a block with Blowfish in ECB mode is below.

\begin{verbatim}
#include <crypt.h>
int main(void)
{ 
    unsigned char pt[8], ct[8], key[8];
    union symmetric_key skey;
    int x;

    /* make up a key */
    for (x = 0; x < 8; x++) pt[x] = key[x] = x;

    /* schedule the key */
    if (blowfish_setup(key, 8, 0, &skey) == CRYPT_ERROR) {
       printf("Setup error: %s\n", crypt_error);
       return -1;
    }    

    /* encrypt the block */
    blowfish_ecb_encrypt(pt, ct, &skey);
 
    /* decrypt the block */
    blowfish_ecb_decrypt(ct, pt, &skey);

    return 0;
}
\end{verbatim}

\section{The Cipher Descriptors}

To facilitate automatic routines an array of cipher descriptors is provided in the array ``cipher\_descriptor''.  An element
of this array has the following format:

\begin{verbatim}
struct _cipher_descriptor {
   char *name;
   int  min_key_length, max_key_length, 
        block_length, default_rounds;
   int  (*setup)      (unsigned char *key, int keylength, 
                       int num_rounds, union symmetric_key *skey);
   void (*ecb_encrypt)(unsigned char *pt, unsigned char *ct, 
                       union symmetric_key *key);
   void (*ecb_decrypt)(unsigned char *ct, unsigned char *pt, 
                       union symmetric_key *key);
   int (*test)        (void);
};
\end{verbatim}

Where ``name'' is the lower case ASCII version of the name.  The fields ``min\_key\_length'', ``max\_key\_length'' and
``block\_length'' are all the number of bytes not bits.  As a good rule of thumb it is assumed that the cipher supports
the min and max key lengths but not always everything in between.  The ``default\_rounds'' field is the default number
of rounds that will be used.

The remaining fields are all pointers to the core functions for each cipher.  The end of the cipher\_descriptor array is
marked when ``name'' equals {\bf NULL}.

As of this release the current cipher\_descriptors elements are

\begin{center}
\begin{tabular}{|c|c|c|c|c|}
     \hline Name & Block Size & Key Range & Rounds \\
     \hline Blowfish, ``blowfish'' & 8 & 8 ... 56 & 16 \\
     \hline RC5-32/12/b, ``rc5'' & 8 & 8 ... 128 & 12 ... 24 \\
     \hline RC6-32/20/b, ``rc6'' & 16 & 8 ... 128 & 20 \\
     \hline SAFER+, ``safer+'' & 16 & 16, 24, 32 & 8, 12, 16 \\
     \hline Serpent, ``serpent'' & 16 & 16 .. 32 & 32 \\
     \hline
\end{tabular}
\end{center}

Note that ``SAFER+'' does not allow the user to override the number of rounds.  The number of rounds varies per 
key length.  There are 8 rounds for 16 byte keys, 12 rounds for 24 byte keys and 16 rounds for 32 byte keys.  You can either
specify the correct number of rounds or zero and let it pick.

To work with the cipher\_descriptor array there is a function ``int find\_cipher({\bf char *}name)'' that will search
for a given name in the array.  It returns negative one if the cipher is not found, otherwise it returns the location
in the array where the cipher was found.  For example, to indirectly setup Blowfish you can also use:

\begin{small}
\begin{verbatim}
cipher_descriptor[find_cipher("blowfish")].setup(key, keylen, rounds, &skey);
\end{verbatim}
\end{small}

A good safety would be to check the return value before accessing the desired function.

\section{CBC and CTR Block Modes}
The library provides simple support routines for handling CBC and CTR encoded messages.  Assuming the mode you want
is XXX there is a structure called ``{\bf struct symmetric\_XXX}'' that will contain the information required to
use that mode.  They have virtually identical setup routines:
\begin{verbatim}
int cbc_start(int cipher, unsigned char *IV, unsigned char *key, 
              int keylen, int num_rounds, struct symmetric_CBC *cbc);

int ctr_start(int cipher, unsigned char *count, unsigned char *key, 
              int keylen, int num_rounds, struct symmetric_CTR *ctr);
\end{verbatim}

In both cases ``cipher'' is the index into the cipher\_descriptor array of the cipher you want to use.  The ``IV'' and
``count'' values are the initialization vectors.  You must fill them yourself and it is assumed they are the same length
as the block size of the cipher you choose.  The parameters ``key'', ``keylen'' and ``num\_rounds'' are the same as in
the XXX\_setup() function call.  The final parameter is a pointer to the structure you want to hold the information for
the mode of operation.

Both routines return {\bf CRYPT\_OK} if the cipher initialized correctly, otherwise they return {\bf CRYPT\_ERROR}.  To 
actually encrypt or decrypt the following routines are provided:
\begin{verbatim}
void cbc_encrypt(unsigned char *pt, unsigned char *ct, 
                 struct symmetric_CBC *cbc);
void cbc_decrypt(unsigned char *ct, unsigned char *pt,
                 struct symmetric_CBC *cbc);

void ctr_encrypt(unsigned char *pt, unsigned char *ct, 
                 int len, struct symmetric_CTR *ctr);
void ctr_decrypt(unsigned char *ct, unsigned char *pt, 
                 int len, struct symmetric_CTR *ctr);
\end{verbatim}

These routines work much like the ECB mode routines.  In the CBC case it always encrypts blocks of a size determined
by the cipher.  This means in CBC mode you should pad your message (with zeroes will work) so the message length
is a multiple of block cipher length.  In CTR mode you can specify the size of each block which means when you reach
the end of the message you can simply pass the value of the remaining length.

To decrypt in either mode you simply setup the CBC or CTR buffer like before (recall you have to fetch the IV value you used)
and use the decrypt routine on all of the blocks.

\chapter{One-Way Hash Functions}
\section{Core Functions}

Like the ciphers there are hash core functions and a universal structure to hold the hash state called 
``{\bf union hash\_state}''.  To initialize hash XXX (where XXX is the name) call:
\begin{verbatim}
void XXX_init(union hash_state *md);
\end{verbatim}

This simply sets up the hash to the default state governed by the specifications of the hash.  To add data to the 
message being hashed call:
\begin{verbatim}
void XXX_process(unsigned char *in, int len, 
                 union hash_state *md);
\end{verbatim}

Essentially all hash messages are virtually infinitely\footnote{All hashes are limited to $2^{32}$ bits} long message which is 
buffered.  You pass pieces of the data at a time to the process function.  To finally get the result of the hash
call:
\begin{verbatim}
void XXX_done(union hash_state *md, 
              unsigned char *out);
\end{verbatim}

This function will finish up the hash and store the result in the ``out'' array.  You must ensure that ``out'' is long
enough for the hash in question.  To test a hash function call:
\begin{verbatim}
int XXX_test(void);
\end{verbatim}

This will return {\bf CRYPTO\_OK} if the hash matches the test vectors, otherwise it returns {\bf CRYPTO\_ERROR}.  An
example snippet that hashes a message with md5 is given below.
\begin{verbatim}
#include <crypt.h>
int main(void)
{
    union hash_state md;
    unsigned char *in = "hello world", out[16];

    /* setup the hash */
    md5_init(&md);

    /* add the message */
    md5_process(in, strlen(in), &md);

    /* get the hash */
    md5_done(&md, out);

    return 0;
}
\end{verbatim}

\section{Hash Descriptors}

Like the set of ciphers the set of hashes have descriptors too.  They are stored in an array called ``hash\_descriptor'' and
are defined by:
\begin{verbatim}
struct _hash_descriptor {
    char *name;
    int hashsize;
    void (*init)   (union hash_state *);
    void (*process)(union hash_state *, 
                    unsigned char *, int);
    void (*done)   (union hash_state *, 
                    unsigned char *);
    int  (*test)   (void);
};
\end{verbatim}

Similarly ``name'' is the name of the hash function in ASCII (all lowercase).  ``hashsize'' is the size of the digest output
in bytes.  The remaining fields are pointers to the functions that do the respective tasks.  There is a function to
search the array as well called ``int find\_hash(char *name)''.  It returns -1 if the hash is not found, otherwise the
position in the descriptor table of the hash.

There are two helper functions as well:
\begin{verbatim}
int hash_memory(int hash, unsigned char *data, 
                int len, unsigned char *dst);

int hash_file(int hash, char *fname, 
              unsigned char *dst);
\end{verbatim}

Both functions return {\bf CRYPT\_OK} on success, otherwise they return {\bf CRYPT\_ERROR}.  The ``hash'' parameter is
the location in the descriptor table of the hash.  The functions are otherwise straightforward.  To perform the above 
hash with md5 the following code could be used:
\begin{verbatim}
hash_memory(find_hash("md5"), "hello world", 11, out);
\end{verbatim}

The following hashes are provided as of this release:
\begin{center}
\begin{tabular}{|c|c|}
      \hline Name & Size \\
      \hline SHA-256, ``sha256'' & 32 \\
      \hline SHA-1, ``sha1'' & 20 \\
      \hline MD5, ``md5'' & 16 \\
      \hline TIGER-192, ``tiger'' & 24 \\
      \hline
\end{tabular}
\end{center}

\chapter{Pseudo-Random Number Generators}
\section{Core Functions}

The library provides an array of core functions for PRNGs as well.  There is a universal structure called 
``{\bf union prng\_state}''.  To initialize a PRNG call:
\begin{verbatim}
int XXX_start(union prng_state *prng);
\end{verbatim}

This will setup the PRNG but not seed it.  Returns {\bf CRYPTO\_ERROR} if there is an error.  In order for the PRNG to be 
cryptographically useful you must give it entropy.  Ideally you'd have some OS level source to tap like in UNIX.  Since 
this is a portable library I leave it upto you to figure this one out.  Nyah Nyah!.  To add entropy to the PRNG call:
\begin{verbatim}
int XXX_add_entropy(unsigned char *in, int len, 
                    union prng_state *prng);
\end{verbatim}

Which returns {\bf CRYPTO\_OK} if the entropy was accepted.  Once you think you have enough entropy you call another
function to put the entropy into action.
\begin{verbatim}
int XXX_ready(union prng_state *prng);
\end{verbatim}

Which returns {\bf CRYPTO\_OK} if it is ready.  Finally to actually read bytes call:
\begin{verbatim}
int XXX_read(unsigned char *out, int len,
             union prng_state *prng);
\end{verbatim}

Which returns the number of bytes read from the PRNG.

\subsection{Remarks}

It is possible to be adding entropy and reading from a PRNG at the same time.  For example, if you first seed the PRNG
and call ready() you can now read from it.  You can also keep adding new entropy to it.  The new entropy will not be used
in the PRNG until ready() is called.  This allows the PRNG to be used and re-seeded at the same time.

No real error checking is guaranteed to see if the entropy is sufficient or if the PRNG is even in a ready state.  Simple
snippet to read 10 bytes from yarrow is below.

\begin{verbatim}
#include <crypt.h>
int main(void)
{
   union prng_state prng;
   unsigned char buf[10];

   /* start it */
   yarrow_start(&prng);

   /* add entropy */
   yarrow_add_entropy("hello world", 11, &prng);
 
   /* ready and read */
   yarrow_ready(&prng);
   yarrow_read(buf, 10, &prng);

   return 0;
}
\end{verbatim}

\section{PRNG Descriptors}

PRNGs have descriptors too (surprised?). Stored in the structure ``prng\_descriptor''.  The format of an element is:
\begin{verbatim}
struct _prng_descriptor {
    char *name;
    int (*start)      (union prng_state *);
    int (*add_entropy)(unsigned char *, int, 
                       union prng_state *);
    int (*ready)      (union prng_state *);
    int (*read)       (unsigned char *, int len, 
                       union prng_state *);
};
\end{verbatim}

There is a ``int find\_prng(char *name)'' function as well.  Returns -1 if the PRNG is not found, otherwise it returns
the position in the prng\_descriptor array.

There are no additional support routines for PRNGs.  Yarrow is the only currently support PRNG available.  It uses by 
default SHA1 and Blowfish but you can change that after calling ``yarrow\_start()''.

\chapter{RSA Routines}
\section{Core Functions}

For RSA routines a single ``struct rsa\_key'' structure is used.  To make a new RSA key call:
\begin{verbatim}
int rsa_make_key(union prng_state *prng, 
                 int wprng, int size, 
                 long e, struct rsa_key *key);
\end{verbatim}

Where ``wprng'' is the index into the PRNG descriptor array.  ``size'' is the size in bytes of the RSA modulus desired.
``e'' is the encryption exponent desired, typical values are 3, 17, 257 and 65537.  I suggest you stick with 65537 since its big
enough to prevent trivial math attacks and not super slow.  ``key'' is where the key is placed.  This routine returns
{\bf CRYPTO\_ERROR} if it fails to make a RSA key.

Todo raw work with the RSA function call:
\begin{verbatim}
int rsa_exptmod(unsigned char *in, int inlen, 
                unsigned char *out, int *outlen, 
                int which, struct rsa_key *key);
\end{verbatim}
This loads the bignum from ``in'' as a little endian word, raises it to either ``e'' or ``d'' and stores the result
in ``out'' and the size of the result in ``outlen''.  You must ensure that ``outlen'' is set to the maximum size of
the output buffer before calling this function.  This is to prevent buffer overruns.  ``which'' is set to 
{\bf PK\_PUBLIC} to use ``e'' (i.e. for encryption) and set to {\bf PK\_PRIVATE} to use ``d'' as the exponent.  This
function returns {\bf CRYPT\_ERROR} on error.

\section{Packet Routines}
The remaining RSA functions are non-standard but should (to the best of my knowledge) be secure if used correctly.  To
encrypt a buffer of memory in a hybrid fashion call:
\begin{verbatim}
int rsa_encrypt(unsigned char *in, int len, 
                unsigned char *out, int *outlen,
                union prng_state *prng, int wprng, 
                int cipher, struct rsa_key *key);
\end{verbatim}
This will encrypt the message with the cipher specified by ``cipher'' under a random key made by a PRNG specified by
``wprng'' and RSA encrypt the symmetric key with ``key''.  This stores all the relevent information in ``out'' and sets
the length in ``outlen''.  You must ensure that ``outlen'' is set to the buffer size before calling this.  This returns
{\bf CRYPT\_ERROR} on error.  To decrypt packets made by this routine call:
\begin{verbatim}
int rsa_decrypt(unsigned char *in, int len, 
                unsigned char *out, int *outlen, 
                struct rsa_key *key);
\end{verbatim}
Which works akin to rsa\_encrypt().  Similarly to sign/verify there are:
\begin{verbatim}
int rsa_sign(unsigned char *in, int inlen, 
             unsigned char *out, int *outlen, int hash, 
             union prng_state *prng, int wprng, struct rsa_key *key);

int rsa_verify(unsigned char *sig, 
               unsigned char *msg, 
               int inlen, int *stat, 
               struct rsa_key *key);
\end{verbatim}

The verify function sets ``stat'' to 1 if it passes or to 0 if it fails.  The ``sig'' parameter is the output of the
rsa\_sign() function and ``msg'' is the original msg that was signed.  To import/export RSA keys as a memory buffer (
e.g. to store them to disk) call:
\begin{verbatim}
int rsa_export(unsigned char *out, int *outlen, 
               int type, struct rsa_key *key);

int rsa_import(unsigned char *in, struct rsa_key *key);
\end{verbatim}

To free the memory used by an RSA key call:
\begin{verbatim}
void rsa_free(struct rsa_key *key);
\end{verbatim}

\backmatter
\appendix
\input{crypt.ind}
\end{document}

\end{document}

\end{document}

\end{document}
